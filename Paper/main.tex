% THIS TEMPLATE IS A WORK IN PROGRESS

\documentclass{article}

\usepackage{hyperref}
\usepackage{fancyhdr}
\usepackage{amsfonts,amsmath}
\usepackage{latexsym}
\usepackage{fullpage}
\usepackage{graphicx}
\usepackage{tikz}
\usepackage{enumitem}
\usepackage{amsthm}
\usepackage{makecell}

%\lhead{\includegraphics[width=0.2\textwidth]{nyush-logo.pdf}}
\fancypagestyle{firstpage}{%
  \lhead{NYU Shanghai}
  \rhead{
  %%%% COMMENT OUT / UNCOMMENT THE LINES BELOW TO FIT WITH YOUR MAJOR(S)
  %\&
  %Data
   Machine Learning 2021}
}

%%%% PROJECT TITLE
\title{Your Wonderful Project Topic}

%%%% NAMES OF ALL THE STUDENTS INVOLVED (first-name last-name)
\author{\href{mailto:author1@nyu.edu}{Author \#1}, \href{mailto:author2@nyu.edu}{Author \#2}}

\date{\vspace{-5ex}} %NO DATE


\begin{document}
\maketitle
\thispagestyle{firstpage}


\begin{abstract}
    
    
    For this project, I am positioning myself as a scouting agency that uses analytics to, among other things, enhance the discovery of talents and help soccer clubs better understand the dynamics (features) that come into play when determining the value, overall and future potential of a player. 
    
    Although the $overall$ rating is the most important value either in the real football market or FIFA games, the official rating criteria is still not published. 
    Our model is to predict the $overall$ rating of players given their several capability values, which use the data classified by players' position and the whole position data.
    Besides, our
    project only uses some of these attributes concerning only player capability according to official guideline.
    
    To demonstrate the benefits of our model, we built various charts and evaluated it. In our experiments, neural networks could ... compared to liner regression.
    
\end{abstract}


\section*{Introduction}
The Fédération Internationale de Football Association (FIFA) will publish the
overall rating and other metrics for soccer players every year. Although the
metrics can affect the overall rating of a player, the exact rating criteria is still
not published. Our project is to predict the overall rating of players given their
capability values.
Several data analyses projects have been designed for this dataset. However,
these projects only focus on evaluating the metrics of players but failed to predict
the overall rating of players. Some projects tried to predict the overall rating
of players, but these projects still lack the comparison among different models
to choose the best one or give any persuasive insights or conclusions. Our project tries different models for rating
prediction and determine their performances to find the best model and give insightful conclusions and opinions.

We implement regression methods in machine learning to this problem.
We have trained our model based on a set of data randomly selected from our
dataset, and then test our model on a stand-out test set also selected from our
dataset. In this way, we create and validate a model predicting the overall
rating of a player.


Our project dataset is FIFA 2019 players attributes dataset, which is collected
from Kaggle. This dataset includes 89 attributes of a soccer player, and our
project only uses some of these attributes concerning only player capability.
Besides, considering that goalkeepers have a totally different set of capability
metrics compared with other players.

We implement linear regression and neural networks in our project. For each model, we use cross validation to find the
best hyperparameter for each model. We compare their performance based
on their $r2 score$ and testing $RMSE$. Finally choose the most accurate model, neural networks. In order to finish this model fitting and predicting process, we plan
to randomly select data from our dataset, and then split the selected data into
training and testing sets to implement cross validation.


\section*{Dataset}

\subsection{Dateset overall}

Our project dataset is FIFA 2019 players attributes dataset, which is collected
from Kaggle. This dataset includes 89 attributes of a soccer player, and our
project only uses some of these attributes concerning only player capability.
Besides, considering that goalkeepers have a totally different set of capability
metrics compared with other players, we will only consider data records of nongoalkeepers in our model.

\subsection{Data processing}

Now we beginning the data processing.
There are 18206 players for which 89 features each are provided. Firstly, we should drop the duplicate entries. 
Then we check the columns (attributes). These columns can be grouped into 6 categories as follows: 

\begin{table}[]
    \centering
    \begin{tabular}{|l|l|}
    \hline
    \makecell[c]{\textit{Category}}     &       \makecell[c]{\textit{Attributes}} \\ \hline
    \textit{Basic Information}          &       \makecell[c]{\textit{ID, Name, Age, Photo, Nationality, Real Face, Height, Weight,} \\
                                                \textit{Body Type, Special, Flag, Position, Club, Club Logo, Work Rate,} \\
                                                \textit{Jersey Number, Weak Foot, Preferred Foot, Skill Moves,} \\
                                                \textit{International Reputation}} \\ \hline
    \textit{Ratings}                    &       \makecell[c]{\textit{Overall, Potential, LS, ST, RS, LW, LF, CF, RF, RW, LAM,} \\
                                                \textit{CAM, RAM, LM, LCM, CM, RCM, RM, LWB, LDM, CDM,} \\
                                                \textit{RDM, RWB, LB, LCB, CB, RCB, RB}} \\ \hline
    \textit{Market Value Related}       &       \makecell[c]{\textit{Value, Wage, Joined, Loaned From, Contract Valid Until,} \\
                                                \textit{Release Clause}} \\ \hline
    \textit{Abilities}                  &       \makecell[c]{\textit{Crossing, Finishing, HeadingAccuracy, ShortPassing, Volleys,} \\
                                                \textit{Dribbling, Curve,FKAccuracy, LongPassing, BallControl,} \\
                                                \textit{Acceleration, SprintSpeed, Agility, Reactions, Balance, ShotPower,} \\
                                                \textit{Jumping, Stamina, Strength, LongShots, Aggression, Interceptions,} \\
                                                \textit{Positioning, Vision, Penalties,Composure, Marking, StandingTackle,} \\ \textit{SlidingTackle}} \\ \hline
    \textit{Goalkeeper Abilities}       &       \makecell[c]{\textit{GKDiving, GKHandling, GKKicking, GKPositioning, GKReflexes}} \\ \hline
    \end{tabular}
    \caption{Columns (attributes) in the original dataset.}
\end{table}

\par Our model only focuses on predicting the overall rating of a player. Thus, we only need to keep those attributes related to the ability of players and drop the other irrelavent attributes. Besides, our model does not consider ratings on different positions for the same player. Thus, we only keep attributes as follows:

\begin{table}[]
    \centering
    \begin{tabular}{|l|l|}
    \hline
    \makecell[c]{\textit{Category}}     &       \makecell[c]{\textit{Attributes}} \\ \hline
    \textit{Basic Information}          &       \makecell[c]{\textit{Age, Position}} \\ \hline
    \textit{Ratings}                    &       \makecell[c]{\textit{Overall} (as target value)} \\ \hline
    \textit{Abilities}                  &       \makecell[c]{\textit{Crossing, Finishing, HeadingAccuracy, ShortPassing, Volleys,} \\
                                                \textit{Dribbling, Curve,FKAccuracy, LongPassing, BallControl,} \\
                                                \textit{Acceleration, SprintSpeed, Agility, Reactions, Balance, ShotPower,} \\
                                                \textit{Jumping, Stamina, Strength, LongShots, Aggression, Interceptions,} \\
                                                \textit{Positioning, Vision, Penalties,Composure, Marking, StandingTackle,} \\ \textit{SlidingTackle}} \\ \hline
    \textit{Goalkeeper Abilities}       &       \makecell[c]{\textit{GKDiving, GKHandling, GKKicking, GKPositioning, GKReflexes}} \\ \hline
    \end{tabular}
    \caption{Columns (attributes) in the dataset for model prediction.}
\end{table}

\par After deleting all the irrelavent attributes, we still have to delete data records with missing values. These records are resulted from , and they could introduce errors into our prediction. The null values in each column are counted and sorted as follows:

\begin{table}[]
\centering
\begin{tabular}{|l|l|l|l|}
\hline
\textit{Index} & \textit{column name}          & \textit{Total missing} & \textit{Percent missing} \\ \hline
\textit{0}     & \textit{Position}             & \textit{60}            & \textit{0.003295}        \\ \hline
\textit{1}     & \textit{GKReflexes}           & \textit{48}            & \textit{0.002636}        \\ \hline
\textit{2}     & \textit{Curve}                & \textit{48}            & \textit{0.002636}        \\ \hline
\textit{3}     & \textit{Agility}              & \textit{48}            & \textit{0.002636}        \\ \hline
\textit{4}     & \textit{SprintSpeed}          & \textit{48}            & \textit{0.002636}        \\ \hline
\textit{5}     & \textit{Acceleration}         & \textit{48}            & \textit{0.002636}        \\ \hline
\textit{6}     & \textit{BallControl}          & \textit{48}            & \textit{0.002636}        \\ \hline
\textit{...}   & \textit{...}                  & \textit{...}           & \textit{...}             \\ \hline
\textit{35}    & \textit{Stamina}              & \textit{48}            & \textit{0.002636}        \\ \hline
\textit{36}    & \textit{Overall}              & \textit{0}             & \textit{0.000000}        \\ \hline
\textit{37}    & \textit{Age}                  & \textit{0}             & \textit{0.000000}        \\ \hline
\end{tabular}
\caption{Missing value percent.}
\end{table}

\par After deleting these missing values, we have 18147 rows left. After that, we still should inspect the data types of each column, and we should ensure the input attributes of our model are all of type $int$ to guarantee that our model works. The only attribute violating this in our model prediction dataset is $Position$. This attribute only indicates the position of a soccer player. Thus, we can divide all records into different categories according to their positions.
\par We mainly divide players into 4 positions: \textbf{Forward}, \textbf{Midfielder}, \textbf{Back}, and \textbf{Goalkeeper}, and 4 CSV files are built for each of these categories to store them separately. Specially, we also built a CSV file to store all data together after our processing above.

\section*{Solution}

The solution section covers all of your model design, algorithms, formulas, findings etc. It explains in detail each contribution, if possible with figures/schematics.




\section*{Results and Discussion}
The results section details your metrics and experiments for the assessment of your solution. It allows you to compare your idea with other approaches you've tested. 

\nocite{*}
\bibliographystyle{IEEEtran}
\bibliography{references}



\end{document}
